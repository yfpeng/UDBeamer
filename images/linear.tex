\documentclass[10pt]{article}

\usepackage{alltt}
\usepackage{amssymb}
\usepackage{amsmath}
%\usepackage{beamerprosper}
\usepackage[english]{babel}
\usepackage{booktabs}
\usepackage{calc}
\usepackage{colortbl}
\usepackage{helvet}
\usepackage{mathptmx}
\usepackage{multirow}
\usepackage{pgf}
\usepackage{smartdiagram}
\usepackage{tabularx}
\usepackage{tikz}
\usepackage{ulem}
\usepackage{xmpmulti}

\usepackage[active,tightpage]{preview}
\PreviewEnvironment{center}
\setlength\PreviewBorder{5pt}

\usetikzlibrary{%
  arrows,%
  automata,%
  calc,%
  trees,%
  positioning,%
  chains,%
  shapes,%
  shapes.arrows,%
  shapes.geometric,%
  shapes.misc,% wg. rounded rectangle
  shapes.symbols,%
  decorations.pathreplacing,%
  decorations.pathmorphing,% /pgf/decoration/random steps | erste Graphik
  matrix,%
  scopes,%
  shadows%
 }
 
\begin{document}

\begin{center}
\tikzstyle{module_text}=[
  draw=none,
  %minimum height=1.5cm,
  minimum width=6cm,
  inner sep=3pt,
  text width=8.5cm,
]

\definecolor{green}{RGB}{0,176,80}
\definecolor{blue}{RGB}{0,176,240}
\definecolor{darkblue}{RGB}{46,117,182}
\definecolor{yellow}{RGB}{255,192,0}
\definecolor{brown}{RGB}{197,90,17}

\newcommand{\subtext}[1]{$_{\text{#1}}$}
\newcommand{\darkbluetext}[1]{\textcolor{darkblue}{\textbf{#1}}}
\newcommand{\greentext}[1]{\textcolor{green}{\textbf{#1}}}

\newcommand*{\xMin}{0}%
\newcommand*{\xMax}{20}%
\newcommand*{\yMin}{-15}%
\newcommand*{\yMax}{0}%

\newcommand{\drawtextbound}[3]{
\draw[color=#1,line width=4pt, rounded corners=.5em] let 
  \p1=#2,
  \p2=#3 in
  (\x1,\y1) -- (\x2,\y1) -- (\x2,\y2)[rounded corners=0] -- (\x1,\y2)[rounded corners=0] -- cycle;
}

\newcommand{\drawarrow}[4] {
\draw[fill=#1,color=#1,line width=4pt] let 
  \p1=#2,
  \p2=#3 in
  (\x1-6.5em,\y1-1pt) -- (\x1-3.5em,\y1-1em) -- (\x2-.5em,\y1-1pt) -- (\x2-.5em,\y2-1pt) -- (\x2-3.5em,\y2-1em) -- (\x1-6.5em,\y2-1pt) -- cycle;
\draw let 
  \p1=#2,
  \p2=#3 in
  ($(\x1-3.5em, \y1-.8em)+.5*(0,\y2)-.5*(0,\y1)$) node[draw=none,text=black]{
  #4
  };
}
\begin{tikzpicture}[x=1cm,y=1cm,>=stealth',node distance=.3cm]

\node at (0,0) [
  module_text,
  anchor=north west
] (sentence) {%
  \begin{tabular}{ll}
  $\bullet$ & \darkbluetext{PKC} \uwave{phosphorylates} \greentext{GAP-43} on serine 41.\\
  $\bullet$ & It was suggested that \darkbluetext{Yak1} \uwave{phosphorylates} \\
   & \greentext{Crf1} to promote its nuclear entry.
  \end{tabular}
};

\node [
  module_text,
  below=of sentence.south west,
  anchor=north west,
] (pattern) {%
  \begin{tabular}{ll}
  $\bullet$ & \darkbluetext{Protein\subtext{kinase}} \uwave{phosphorylates}\\
  $\bullet$ & \uwave{phosphorylates} \greentext{Protein\subtext{substrate}}
  \end{tabular}
};

\node [
  module_text,
  below=of pattern.south west,
  anchor=north west,
] (result) {%
  \begin{tabular}{lp{5em}l}
  $\quad$ & \darkbluetext{Kinase} & \greentext{Substrate}\\
  \cmidrule(rr){2-3}
  & PKC & GAP-43\\
  & Yak1 & Crf1
  \end{tabular}
};

\drawtextbound{blue}{(sentence.north west)}{(sentence.south east)};
\drawarrow{blue}{(sentence.north west)}{(sentence.south west)}{
  \begin{tabular}{l}
  \textbf{Sentence}
  \end{tabular}%
};

\drawtextbound{yellow}{(pattern.north west)}{(pattern.south east)};
\drawarrow{yellow}{(pattern.north west)}{(pattern.south west)}{
  \begin{tabular}{l}
  \textbf{Pattern}
  \end{tabular}%
};

\drawtextbound{brown}{(result.north west)}{(result.south east)};
\drawarrow{brown}{(result.north west)}{(result.south west)}{
  \begin{tabular}{c}
  \footnotesize
  \textbf{Phosphorylation}\\
  \textbf{Results}
  \end{tabular}%
};

%\foreach \i in {\xMin,...,\xMax} {
%	\draw [thin,gray] node [below] at (\i,\yMin) {$\i$};
%}
%\foreach \i in {\yMin,...,\yMax} {
%  \draw [thin,gray] node [left] at (\xMin,\i) {$\i$};
%}
%
%\draw[step=.5,gray,thin] (\xMin,\yMin) grid (\xMax,\yMax);

\end{tikzpicture}
\end{center}

%----------------------------------------

\begin{center}

\begin{tikzpicture}

\definecolor{color1}{RGB}{255,187,0}
\definecolor{color2}{RGB}{150,211,51}
\definecolor{color3}{RGB}{0,195,201}

% center point, color left text, right text
\newcommand{\oneblock}[4]{
  \draw #1 node {
  \begin{tikzpicture}[node distance=0em]
  \node[draw=#2,
        rounded corners,
        line width=4pt,
        inner sep=4pt,
        text justified,
        text width=.7\textwidth,
        minimum height=4em](a) {#4};
  \draw[fill=#2,draw=none]
        let \p1=(a.north west), 
            \p2=(a.south west),
            \p3=(\x1+5pt,\y1),
            \p4=(\x2+5pt,\y2) in
        (\p3)--(\x3-3em,\y3-1.5em)--(\x3-6em,\y3)
        --(\x4-6em,\y4)--(\x4-3em,\y4-1.5em)--(\p4)--cycle;
  \node[right =of a.west,
        xshift=-3em+5pt, 
        yshift=-1em, 
        anchor=center, 
        align=center] {#3};
  \end{tikzpicture}
  };
}

\newcommand*{\entity}[1]{\textcolor{blue}{#1}}
\oneblock{(0em,0em)}{color1}{Training set}{
    $\bullet$ ~ \entity{PKC} \uwave{phosphorylates} \entity{GAP-43} on serine 41.\\
    $\bullet$ ~ \entity{JNK} \uwave{phosphorylates} \entity{BAD} at threonine 201.\\
    $\bullet$ ~ $\cdots$
}
\oneblock{(0em,-5em)}{color2}{Learned\\rules}{
	\entity{A} \uwave{phosphorylates} \entity{B}, \\
    when \entity{A} and \entity{B} are proteins
}
\oneblock{(0em,-10em)}{color3}{New Data}{
	It was suggested that \entity{Yak1} \uwave{phosphorylates} \entity{Crf1} to promote its nuclear entry.
}
\end{tikzpicture}
\end{center}

\end{document}