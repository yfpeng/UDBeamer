\documentclass[10pt]{article}

\usepackage[rgb]{xcolor}

\usepackage{alltt}
\usepackage{amssymb}
\usepackage{amsmath}
%\usepackage{beamerprosper}
\usepackage[english]{babel}
\usepackage{booktabs}
\usepackage{calc}
\usepackage{colortbl}
\usepackage{helvet}
\usepackage{mathptmx}
\usepackage{multirow}
\usepackage{pgf}
\usepackage{smartdiagram}
\usepackage{tabularx}
\usepackage{tikz}
\usepackage{ulem}
\usepackage{xmpmulti}

\usepackage[active,tightpage]{preview}
\PreviewEnvironment{center}
\setlength\PreviewBorder{5pt}

\pgfdeclarelayer{background}
\pgfdeclarelayer{foreground}
\pgfsetlayers{background,main,foreground}

\usetikzlibrary{%
  arrows,%
  automata,%
  calc,%
  trees,%
  positioning,%
  chains,%
  shapes,%
  shapes.arrows,%
  shapes.geometric,%
  shapes.misc,% wg. rounded rectangle
  shapes.symbols,%
  decorations.pathreplacing,%
  decorations.pathmorphing,% /pgf/decoration/random steps | erste Graphik
  matrix,%
  scopes,%
  shadows%
 }
 
\begin{document}

\begin{center}
\begin{tikzpicture}
	\footnotesize
	% Create the background in the circle, by drawing several slices
	% each with a constant color given by the angle (which is converted
	% to a color usin the hue, saturation and brightness color space).
	\foreach \x in {0,0.0416,...,1} {
	  \definecolor{currentcolor}{hsb}{\x, 1, 1}
	  \draw[draw=none, fill=currentcolor]
	    (-360*\x+88-7.5:0) -- (-360*\x+88-7.5:2.7)
	    arc (-360*\x+88-7.5:-360*\x+88+7.5:2.7) -- cycle;
	}
	% Draw a circle with markings along the perimeter, indicating which angles
	% the hue function connects to certain colors.
	\draw (0, 0) circle (2.7);
	\foreach \x  in {0, 30, ..., 330}
	  \draw (-\x+90:2.8) -- (-\x+90:2.7) (-\x+90:3); % node {$\x^\circ$};
	
	% Add labels with names of the primary and secondary colors.
	\foreach \x/\text in {%
	  0/\begin{tabular}{c}Red\\255-0-0\end{tabular},
	  60/\begin{tabular}{c}Yellow\\255-255-0\end{tabular},
	  120/\begin{tabular}{c}Green\\0-255-0\end{tabular},
	  180/\begin{tabular}{c}Cyan\\0-255-255\end{tabular},
	  240/\begin{tabular}{c}Blue\\0-0-255\end{tabular},
	  300/\begin{tabular}{c}Magenta\\255-0-255\end{tabular},
	  %
	  30/\begin{tabular}{c}Orange\\255-125-0\end{tabular},
	  90/\begin{tabular}{c}Spring Green\\125-255-0\end{tabular},
	  150/\begin{tabular}{c}Turquoise\\0-255-125\end{tabular},
	  210/\begin{tabular}{c}Ocean\\0-125-255\end{tabular},
	  270/\begin{tabular}{c}Violet\\125-0-255\end{tabular},
	  330/\begin{tabular}{c}Raspberry\\255-0-125\end{tabular},
	}
	\draw (-\x+90:3.3) node {\text};
\end{tikzpicture}
\end{center}

\end{document}